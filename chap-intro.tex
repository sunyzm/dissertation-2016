\chapter{Introduction}

\section{Problem Statement}

With the advent of ever more advanced and affordable 3D data acquisition
technologies, digitalized 3D shapes have become ubiquitous in our life,
playing fundamental roles in numerous fields and applications, including
computer-aided design, entertainment, medical research, etc.

In order to effectively encode 3D geometric data and facilitate different analysis
and processing tasks, many different shape representations have been developed,
including point clouds, polygon meshes, spline surfaces, voxels, etc. Choosing the
most appropriate representation generally depends on the specific data source and
geometric task at hand. Of all these representations, polygon meshes, especially
triangle meshes, are perhaps the most widely used. Thanks to their conceptual
simplicity, triangle meshes are flexible, versatile, and highly efficient for batch
processing and rendering. In principle, a triangle mesh $\mathcal{M}$ can be deemed
as the geometric embedding of a graph structure $G$ into $\mathbb{R}^3$. In another word,
$\mathcal{M}$ can be decomposed into the topological component, namely the underlying
graph structure $G$, and the geometric component, i.e., the vertex coordinate function
$\mathbf{P}$ which assigns a 3D position $p_i\in\mathbb{R}^3$ to each vertex $v_i$ of
the graph.

%It is natural
%and straightforward to represent and process shape geometry directly on the
%spatial domain, but for many shape modeling applications, Fourier domain
%affords more powerful and convenient representations.

From the point of view of signal processing, the vertex coordinate function of a mesh
is essentially a vector-valued discrete signal defined on the domain of mesh vertices,
just like an image is a color-valued signal defined on 2D grids. The enormous success
of applications of digital signal processing with other types of signals, such as audio
and images, has long motivated people to employ and adapt tools and techniques in signal
processing to tackle geometry processing problems. For instance, in one of the pioneering
works~\cite{Taubin1995} on geometric signal processing, Taubin showed that the problem of
surface smoothing can be carried out via low-pass filtering of discrete surface signals.

One of the key aspects of signal processing to represent the signal in a transform domain
by decomposing it as the linear combination of a suitable choice of basis vectors.
Analyzing the coefficient representation with respect to the transform basis can often reveal
important properties of the original signal, and many operations on the signal can be
accomplished or simplified by modifying the transform coefficients. In Euclidean space,
the most fundamental signal transform is the well known Fourier transform, which converts
signals from time/space domain to frequency domain.

On the general manifold domain, the natural equivalent to the classic Fourier basis is the
manifold harmonic basis (MHB)~\cite{Vallet2008}, also known as the Laplacian eigenbasis,
which is actually the set of eigenfunctions of the manifold's Laplace-Beltrami operator.
According to spectral theory, the Laplacian eigenfunctions form an orthonormal and complete
basis, thus inducing the manifold harmonic transform (MHT) which affords space-frequency
decomposition on the manifold. In the discrete setting, MHB refers to the eigenvectors of the
mesh/graph Laplacian matrix. Unlike classic Fourier basis functions which are simply fixed
sinusoids, the mesh Laplacian eigenbasis would change with the connectivity, geometry, and the
type of Laplacian matrix that is adopted~\cite{Zhang:2010:CGF}. As a result, the mesh Laplacian
eigenbasis encode substantial structural and geometrical information of the original meshes. A
great number of shape modeling and analysis applications have been built directly upon the MHB,
including compression~\cite{Karni2000}, segmentation~\cite{Liu2007},
deformation~\cite{Rong2008}, remeshing~\cite{dong2006spectral},
paramterization~\cite{Zhou2004}, etc.

Myriads of successful applications notwithstanding, the manifold harmonic basis, as well as the
classic Fourier basis, also have their drawbacks. Primarily, all manifold harmonic basis
functions have global support over the entire manifold, hence they are suitable for capturing
the global shape but are not very efficient in representing local features. To more concisely
and accurately represent signals with different properties, wavelet representations and
sparsity-seeking methods have gained great momentum recently in the field of signal
processing, and we would like to adapt these two concepts to the manifold/mesh domain for shape
modeling and analysis.

The basic idea of sparse representation is to use as few as possible elementary functions chosen in a dictionary
to decompose or approximate a signal, based on the intuition that a meaningful high-dimensional signal probably
possesses a low-dimensional intrinsic structure, which can be captured by a sparse coefficient representation with
respect to a suitable dictionary. Given a redundant dictionary, we can compute a sparse decomposition or approximation
of the original signal by using various sparse optimization algorithms. The obtained sparse representation provides not
only a more concise description of the original signal, which can be utilized for signal compression, but also, in many cases,
a more precise one. Thus, computing the sparse representation can help manifest the essential parts of a signal,
facilitating a wide range of applications such as pattern recognition, noise reduction, source separation, signal restoration,
and compressed sensing.

The efficacy of sparsity-based methods depends on the the selection of dictionaries. Generally, the more ``expressive''
the dictionary is, the fewer elementary functions are needed to faithfully reconstruct the original signal, as there
are more ``words'' available to express the information. Hence, instead of selecting a complete and orthogonal basis
like the Fourier basis as the dictionary, it is often desirable to construct a redundant, overcomplete dictionary containing
a rich variety of elementary signals to achieve enhanced expressive power.

A natural choice to construct redundant dictionary is to use wavelets, which are wave-like oscillating functions
localized both in space and in frequency domain. A family of wavelets can be generated by scaling and translating a
single mother wavelet function, forming a rich dictionary of elementary signals of different frequencies and centered
at different locations. However, defining wavelets and wavelet transform on the mesh or manifold domain have always been
a challenging problem, since there is no intuitive way to define scaling on irregular mesh grids. Recently, Hammond et al.
proposed the spectral graph wavelets (SGWs) and spectral graph wavelet transform (SGWT)~\cite{Hammond2011}
which define wavelet functions with the spectral graph basis and performs scaling in the Fourier domain. The SGWs define
a overcomplete wavelet frame and properties such as multiscale and spatial-locality of SGWs have proved to be valuable in
a variety of data analysis applications.

In this dissertation, we present solutions combining graph-based spectral representations and
sparsity-seeking techniques to a series of fundamental problems in shape analysis and geometry
processing, including mesh compression, surface inpainting, feature description, and shape
correspondence. Through various experiments, we demonstrate the competitive performance of
our proposed methods and the great potential of spectral sparse representations. Moreover,
we also outline a few new research directions, including shape editing, shape retrieval,
and sparse representation learning, which lead towards our future plans.

\section{Contributions}

\begin{figure}
  \centering
  \includegraphics[width=0.8\linewidth]{hierarchy}
  \caption[Hierarchy of this dissertation]
  {Hierarchy of this dissertation.}
  \label{fig:thesis_hierarchy}
\end{figure}

Fig.~\ref{fig:thesis_hierarchy} highlights the hierarchy of this dissertation. \emph{TODO: more explanation}.\\

Specifically, the contributions of this dissertations are summarized as follows:

\begin{itemize}
\item We present an innovative approach to 3D mesh compression using spectral graph
wavelets as dictionary to encode mesh geometry. In contrast to
Laplacian eigenbasis, the spectral graph wavelets are locally
defined at individual vertices and can better capture local shape
information in a more accurate way. Nonetheless, the multi-scale
spectral graph wavelets form a redundant dictionary as shape bases,
so we formulate the compression of 3D shape as a sparse
approximation problem that can be readily handled by
algorithms such as orthogonal matching pursuit. Various experiments
demonstrate that our method are superior to the existing spectral
mesh compression methods.

\item  We devise a new algorithm for completing surface with
  missing geometry and topology founded upon the theory and techniques
  of sparse signal recovery. We find that for many shapes the vertex coordinate function
  can be well approximated by a very sparse coefficient representation with respect
  to the dictionary comprising its Laplacian eigenbases, and it is then possible to
  recover this sparse representation from partial measurements of the original shape.
  Taking advantage of the sparsity cue, we advocate a novel
  variational approach for surface inpainting, integrating data
  fidelity constraints on the shape domain with coefficient sparsity
  constraints on the transformed domain. Because of the powerful
  properties of Laplacian eigenbases, the inpainting results of our
  method tend to be smooth and globally coherent with the remaining
  shape. We demonstrate the performance of our new method via various
  examples in geometry restoration, shape repair, and hole filling.

\item We propose a new kind of shape descriptors built upon
  power spectral graph wavelets (SGWs) that are both multi-scale
  and multi-level in nature, consisting of both local (differential)
  and global (integral) information, for the characterization of
  user-specified feature regions. In addition, we develop a generalized
  feature detection framework to facilitate a host of graphics applications,
  including partial matching without point-wise correspondence, coarse-to-fine
  recognition, model recognition, etc. By conducting extensive
  experiments and making comprehensive comparisons with the current
  state-of-the-art, our framework has exhibited many attractive
  advantages such as being geometry-aware, versatile, robust, discriminative, and isometry-invariant.

\item \emph{TODO: tensor-based region feature matching.}

\end{itemize}

\section{Dissertation Organization}
The remainder of this proposal is organized as follows.
In Chapter~2, we briefly review harmonic analysis in the manifold and mesh
domain, including the theories and applications of manifold harmonic basis
and spectral graph wavelets; we also give an overview of the concept of
sparse and redundant representation as well as related computational methods.
In Chapter~3, we present an innovative approach to 3D mesh compression
using spectral graph wavelets as dictionary to encode mesh geometry; in
contrast to Laplacian eigenbasis, the spectral graph wavelets are locally
defined at individual vertices and can better capture local shape
information in a more accurate way.
In Chapter~4, we devise a new algorithm for completing surface with missing geometry
and topology founded upon the theory and techniques of sparse signal recovery
using the manifold harmonic basis as dictionary.
In Chapter~5, we propose a new kind of shape descriptors built upon
spectral graph wavelets for the characterization of user-specified feature regions
and develop a generalized feature detection framework.
In Chapter~6, we present a new algorithm for feature-driven shape matching based on
our generalized region descriptor and high-order graph matching. 
Finally, we conclude this dissertation and outline a few future research directions
in Chapter~7.  