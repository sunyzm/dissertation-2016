\chapter{Introduction}

With the advent of ever more advanced and affordable 3D data acquisition
technologies, digitalized 3D shapes have become ubiquitous in our daily
life and played indispensable roles in numerous fields and applications,
including computer-aided design, entertainment industry, medical research, etc.
Although it is straightforward to represent and manipulate shape geometry
directly in the spatial domain, in recent years there has been a growing trend
towards spectral shape analysis, leveraging the eigen-structures of various mesh
operators, especially the Laplacian operator. In this chapter, we give an overview of
the main theme of this dissertation: spectral representations and sparsity-driven
algorithms for shape modeling and analysis.

\section{Problem Statement}

Spectral analysis refers to the analysis in terms of the
eigenvalues and eigenfunctions of certain linear operator. In the context
of geometry processing and analysis, the most popular spectrum is derived
from the eigen-decomposition of the discrete Laplacian operator, and a great
diversity of spectral methods have been developed for a wide range of shape modeling
problems, including compression~\cite{Karni2000}, segmentation~\cite{Liu2007},
deformation~\cite{Rong2008}, remeshing~\cite{dong2006spectral},
paramterization~\cite{Zhou2004}, shape indexing~\cite{Reuter:2006:CAD, Rustamov:2007:LEF},
and retrieval~\cite{Lavoue:2012}. In these applications, we sometimes
directly use the Laplacian eigenvectors as Fourier-like harmonic basis to
encode shape geometry, and sometimes use the eigenvectors and eigenvalues to
construct more sophisticated shape descriptors or measures.

From the point of view of signal processing, the vertex coordinate function of
a mesh is essentially a vector-valued discrete signal defined on the domain of
mesh vertices, just like an image is a color-valued signal defined on 2D grids.
The enormous success of applications of digital signal processing with other types
of signals, such as audio and images, has long motivated people to employ and
adapt techniques in classic signal processing to tackle geometry
processing problems in the mesh domain. For instance, in one of the pioneering
works~\cite{Taubin1995} on geometric signal processing, Taubin showed that the
process of surface smoothing can be carried out by applying the graph Laplacian
operator to the mesh coordinates, which is equivalent to low-pass filtering
of discrete signals defined on vertices.

One key aspect of signal processing is to represent the signal in a transform
domain by decomposing it as the linear combination of a suitable choice of basis
vectors. Analyzing the coefficient representation with respect to the transform
basis can often reveal important properties of the original signal, and many
shape operations that are difficult to achieve directly on the mesh domain
can be easily accomplished in the transform domain by manipulating the
transform coefficients. In Euclidean space, the most fundamental signal transform
is the well known Fourier transform, which converts signals from time/space domain to
frequency domain with multi-scale sinusoids as basis.

In the general manifold domain, the natural equivalent to the classic Fourier basis is
the manifold harmonic basis (MHB)~\cite{Vallet2008}, which is actually the set
of eigenfunctions of the manifold's Laplace-Beltrami operator.
According to the spectral theory, the Laplacian eigenfunctions form an orthonormal and complete
basis of functions defined on the manifold, i.e., the Laplacian eigenbasis, thus inducing the
manifold harmonic transform (MHT) which affords space-frequency decomposition on the manifold, similar
to the Fourier transform in regular domains. In the discrete setting, MHB refers to the eigenvectors
of the mesh/graph Laplacian matrix.

Unlike classic Fourier basis functions which are simply fixed sinusoids, the manifold harmonic basis
differ with the connectivity, geometry, and the type of Laplacian operator that is adopted~\cite{Zhang:2010:CGF}.
As a result, the mesh Laplacian eigenvectors and eigenvalues actually encode substantial topological and
geometric information and can help characterize the global shape property and reveal intrinsic structure of the
original mesh. This lends to the popularity of spectral methods in
the area of geometry processing and analysis.

Myriads of successful applications notwithstanding, there are certain limitations in employing the Laplacian
eigenpairs directly for shape analysis. Primarily, the Laplacian eigenvalues and eigenvectors are determined
by the global Laplacian matrix and thus encode information of the entire shape. Hence, they are more suitable
for representing the overall shape but are not very effective in encoding local details. In addition, the Laplacian
eigenpairs are not very stable across shapes and direct comparisons become unreliable after the first few eigenpairs.
To more effectively describing the properties of local regions and the pair-wise relations between regions
in a shape, people have developed a series of sophisticated spectral representations building upon the
Laplacian eigenpairs, including various forms of kernels and distances.

In recent years, the focus of harmonic analysis has been moving from orthogonal basis with
minimum size like the Fourier basis to richer, more expressive dictionaries with many redundant
atoms such as wavelets, and from simple determinant time-frequency transform to coefficient
decomposition based on sparsity-driven optimization. Comparing with Fourier-based signal processing,
richer dictionaries are more flexible and have greater expressive power. With appropriate
sparsity-seeking algorithm, the obtained representation provides
not only a more concise description of the original signal, but also, in many cases, a more precise one.
Thus, computing the sparse representation can help manifest the essential components of a signal,
facilitating a diversity of applications such as signal compression, pattern recognition,
noise reduction, source separation, and signal restoration.

Wavelets and sparse representation have gained great momentum in the signal processing
community~\cite{Mallat2008}, but most successful applications focus on regular domains
such as images and audio. Currently there is very few existing study taking advantages of
graph wavelets and sparsity-seeking techniques to solve problems in shape analysis and processing.

\section{Contributions}

\begin{figure}
  \centering
  \includegraphics[width=0.75\linewidth]{pipeline}
  \caption[Hierarchy of this dissertation]
  {Conceptal hierarchy of this dissertation.}
  \label{fig:thesis_hierarchy}
\end{figure}

The basic idea of sparse representation is to use as few as possible elementary functions chosen
in a dictionary to decompose or approximate a signal, based on the intuition that a meaningful
high-dimensional signal probably possesses a low-dimensional intrinsic structure, which can be
captured by a sparse coefficient representation with respect to a suitable dictionary. Given a
redundant dictionary, we can compute a sparse decomposition or approximation of the original signal
by using various sparse optimization algorithms. The obtained sparse representation provides not
only a more concise description of the original signal, which can be utilized for signal compression,
but also, in many cases, a more precise one.

The efficacy of sparsity-based methods depends on the the selection of dictionaries.
Generally, the more ``expressive'' the dictionary is, the fewer elementary functions
are needed to faithfully reconstruct the original signal, as there are more ``words''
available to express the information. Hence, instead of selecting a complete and orthogonal basis
like the Fourier basis as the dictionary, it is often desirable to construct a redundant,etrieval algorithms based on spectral
representations. Features are encoded with heat kernel signatures
or our SGW-derived feature descriptors. For coarse matching, we adopt
tensor-based high-order graph matching to maximizes the
geometric compatibility between features tuples, and for dense matching,
we present a hierarchical shape registration algorithm, generating
correspondence in multiples levels in a coarse-to-fine manner. We also
propose the novel Bag-of-Feature-Graph (BoFG) descriptor for shape retrieval
For each geometric word in the vocabulary, BoFG constructs a graph that records
spatial relations of all feature pairs in the shape, weighted by their
similarities to this word. The BoFG descriptor significantly reducing the number
of pointed required for computing distributions in comparisons with more traditional
Bag-of-Features descriptors.

A natural choice for constructing redundant dictionary is to use wavelets. A family
of wavelets can be generated by scaling and translating a single mother wavelet function,
forming a rich dictionary of elementary signals of different frequencies and centered
at different locations. However, defining wavelets and wavelet transform on mesh or
manifold domain have always been a challenging problem, since there is no intuitive way
to define scaling on irregular mesh grids. Recently, Hammond et al. proposed the spectral
graph wavelets (SGWs) and spectral graph wavelet transform (SGWT)~\cite{Hammond2011}
, in which the wavelet functions are defined with the spectral graph basis and scaling is carried
out in the Fourier domain. The SGWs constitute an overcomplete wavelet frame, whose properties
such as multiscale and spatial-locality of SGWs have proved to be valuable in a variety of data
analysis applications.

In addition to acting as overcomplete basis for shape signals, the spectral
graph wavelets also captures valuable geometric information of the original
shape in a multiscale and spatially-localized way, thus are very suitable to
constitute the building blocks of local shape descriptors.

In this dissertation, we present solutions combining graph-based spectral
representations, especially spectral graph wavelets and heat kernels, and
sparsity-seeking techniques to a series of fundamental problems in shape
analysis and geometry processing, including mesh compression, surface
inpainting, feature description, shape correspondence, and shape retrieval.
Through various experiments, we demonstrate the competitive performance of
our proposed methods and the great potential of spectral sparse
representations. Fig.~\ref{fig:thesis_hierarchy} highlights the hierarchy
of this dissertation. Specifically, the contributions of this dissertations
are summarized as follows:

\begin{itemize}
\item We present an innovative approach to 3D mesh compression using spectral graph
wavelets as dictionary to encode mesh geometry. In contrast to
Laplacian eigenbasis, the spectral graph wavelets are locally
defined at individual vertices and can better capture local shape
information in a more accurate way. Nonetheless, the multi-scale
spectral graph wavelets form a redundant dictionary as shape bases,
so we formulate the compression of 3D shape as a sparse
approximation problem that can be readily handled by
algorithms such as orthogonal matching pursuit. Various experiments
demonstrate that our method are superior to the existing spectral
mesh compression methods.

\item  We devise a new algorithm for completing surface with
  missing geometry and topology founded upon the theory and techniques
  of sparse signal recovery. We find that for many shapes the vertex coordinate function
  can be well approximated by a very sparse coefficient representation with respect
  to the dictionary comprising its Laplacian eigenbases, and it is then possible to
  recover this sparse representation from partial measurements of the original shape.
  Taking advantage of the sparsity cue, we advocate a novel
  variational approach for surface inpainting, integrating data
  fidelity constraints on the shape domain with coefficient sparsity
  constraints on the transformed domain. Because of the powerful
  properties of Laplacian eigenbases, the inpainting results of our
  method tend to be smooth and globally coherent with the remaining
  shape. We demonstrate the performance of our new method via various
  examples in geometry restoration, shape repair, and hole filling.

\item We propose a new kind of shape feature descriptors built upon the
  coefficients of spectral graph wavelets and biharmonic distance fields.
  Our novel descriptors are both multi-scale and multi-level in nature,
  effectively encoding both local and global information
  for the characterization of user-specified feature regions.
  Via extensive experiments and comprehensive comparisons with the state-of-the-art,
  our descriptor has exhibited many attractive advantages such as being geometry-aware,
  versatile, robust, discriminative, and isometry-invariant.

\item We develop effective feature-driven articulated shape
correspondence and retrieval algorithms based on spectral
representations. Features are encoded with heat kernel signatures
or our SGW-derived feature descriptors. For coarse matching, we adopt
tensor-based high-order graph matching to maximizes the
geometric compatibility between features tuples, and for dense matching,
we present a hierarchical shape registration algorithm, generating
correspondence in multiples levels in a coarse-to-fine manner. We also
propose the novel Bag-of-Feature-Graph (BoFG) descriptor for shape retrieval
For each geometric word in the vocabulary, BoFG constructs a graph that records
spatial relations of all feature pairs in the shape, weighted by their
similarities to this word. The BoFG descriptor significantly reducing the number
of pointed required for computing distributions in comparisons with more traditional
Bag-of-Features descriptors.

\end{itemize}

\section{Dissertation Organization}
The remainder of this proposal is organized as follows.
In Chapter~2, we briefly review harmonic analysis in the manifold and mesh
domain, including the theories and applications of manifold harmonic basis
and spectral graph wavelets; we also give an overview of the concept of
sparse and redundant representation as well as related computational methods.
In Chapter~3, we present an innovative approach to 3D mesh compression
using spectral graph wavelets as dictionary to encode mesh geometry; in
contrast to Laplacian eigenbasis, the spectral graph wavelets are locally
defined at individual vertices and can better capture local shape
information in a more accurate way.
In Chapter~4, we devise a new algorithm for completing surface with missing geometry
and topology founded upon the theory and techniques of sparse signal recovery
using the manifold harmonic basis as dictionary.
In Chapter~5, we propose a new kind of shape descriptors built upon
spectral graph wavelets for the characterization of user-specified feature regions
and develop a generalized feature detection framework.
In Chapter~6, we present a new algorithm for feature-driven shape matching based on
our generalized region descriptor and high-order graph matching.
Finally, we conclude this dissertation with discussions and outline a few future
research directions in Chapter~7.   