\chapter{Generalized Feature Description and Detection}

\section{Introduction}

Studies in feature abstraction and analysis have been
gaining momentum because they can assist numerous downstream graphics
tasks and applications such as shape recognition, segmentation,
analysis, understanding, etc
~\cite{Bronstein2011,Kin-ChungAu:2012,Song:2014:MSV}. Influenced
by the newly-arisen concept of high-level representations in computer
vision, which are based on object-wise components, now much more
attention has been directed towards region-wise feature analysis in
computer graphics. In this work, we advocate a new region-based and
user-specified type of feature as well as a novel
graph-wavelet-inspired multi-scale and multi-level descriptor, and they
jointly enable our feature detection framework that can further
facilitate various practical applications.

Conventional feature descriptors are usually constructed by considering
the discontinuities of certain differential attributes at different
orders (e.g., the second-order attribute like surface curvature) that
naturally afford their discriminative power in characterizing point
features, line/curve features, small patch-based features with regular
boundaries, etc. Such descriptions have facilitated point/patch-based
recognition, point-wise correspondence, and curvature-based saliency
detection with great success. However, for more complex applications
such as modeling by example~\cite{Funkhouser:2004}, model
composition~\cite{Kreavoy:2007}, and key component
analysis~\cite{Sipiran:2012}, the aforementioned conventional features
are usually too localized to capture the multi-scale neighboring
information, and it is desirable to have a flexible, region-wise feature
description. Furthermore, in many real-world settings, shape data may be
degraded due to acquisition imperfections and noises, necessitating the
use of region descriptors which tend to be much more robust.

Existing works related to region-wise analysis include partial matching,
shape correspondence, saliency extraction, etc. Boundaries of the regions
in question are usually confined to regular but non-adaptive
shapes~\cite{Kazhdan:2003,Mortara:03,Gatzke:2005,Kokkinos:2012}, thus
neighboring and in-between geometry information may not be fully captured.
As for region description, trending measures include the distributions of
various types of point
descriptors~\cite{Osada:2002,Ben-Chen:2008,Liu:2006} and the global
analysis of the regions based on spectral
decomposition~\cite{Hu2009,Lavoue:2012}. Some regional measures are
not discriminative enough to solely characterize the regions in question,
for which post-processing like geometric
hashing~\cite{Yehezkel:1988} or random sample
consensus~\cite{Fischler:1981} is required. Nevertheless, these settings
are not hierarchical enough to characterize the focal regions. On the
other hand, multi-scale shape analysis
methods~\cite{Rustamov2011,Sun:2009:CGF}, in spite of their great
descriptive power, have not yet been employed to construct regional
descriptions. These insights inspire us to propose a more comprehensive
and stable type of shape description that can encode the region of
interest with high discriminative power and efficiency.


\begin{figure}[!to]
\begin{center}
\includegraphics[width=1\linewidth]{Fig_1}
\end{center}
\caption[Pipeline of our generalized feature detection framework.]
{Pipeline of our feature detection and description framework.}
\label{pipeline}
\end{figure}

In this chapter, we propose a local-to-global shape feature via user
specification, introduce an informative region descriptor, and then
present a shape feature detection framework to facilitate a host of
graphics applications. The proposed shape feature extends the
definition of conventional features to a region-wise manner in a
user-specified way (as highlighted in Fig.~\ref{pipeline}). To encode
user-specified features, we proactively seek an informative regional
descriptor constructed in a multi-scale and multi-level way. Our
descriptor takes advantage of the bi-harmonic distance field and the
SGWs. The stability and robustness of bi-harmonic distance field
guarantee the technically-sound foundation for the whole descriptor.
SGWs naturally accommodate local and global geometry with a
multi-scale solution, and such solution is consistent across multiple
levels. We devise a new statistical method based on the decomposition
coefficients of the shape signal, enabling the joint analysis of the
underlying geometry together with different shape signals. In order
to comprehensively characterize the shape features, we also
incorporate the contour-centered geometric statistics into our
descriptor. All of these enable our feature detection framework as
shown in Fig.~\ref{pipeline}. We quantify each model's regions of
interest around central points (or point samples) according to the
user-specified feature scope on the query model. Then descriptors are
constructed on candidate regions across different models, which is
equivalent to transform each region into the high-dimensional feature
space. After the region-wise feature space is constructed, various
analytical tasks can be performed. The primary contributions of this
work can be summarized as follows:

\begin{itemize}
\item We propose to define a generalized shape feature type via user
    specification, which is geometry-aware and is the organic coupling
    of local and global description. Also, it is a fundamental tool
    that can help unite different types of graphics applications.

\item Our region-based descriptor is primarily built upon the SGWs
    that are both multi-scale and multi-level in nature, elegantly
    integrating both local (differential) and global (integral)
    information. We also introduce the contour-centered geometric
    statistics to enhance the descriptor's discriminative power.

\item We develop a feature detection framework, which can integrate
    different types of state-of-the-art region descriptors and further
    facilitate widespread graphics applications including partial
    matching, coarse-to-fine recognition, model recognition, etc.

\end{itemize}

\section{Related Work}
\label{sec:RW}

This section will briefly review prior research related to region
analysis and the latest progresses on SGWs.

\textbf{Region-wise Description and Detection.} We first review how
current methods define boundaries for given regions. Broadly speaking,
they characterize a local neighborhood around a central point in two
ways. The first is based on Euclidean distance, such as
spheres~\cite{Kazhdan:2003}, blowing bubbles~\cite{Mortara:03},
rings~\cite{Gatzke:2005}, shape context~\cite{Kokkinos:2012} or priori
decompositions~\cite{Gal2006,Itskovich2011}. The second is to use
geodesic distance like geodesic fans~\cite{Zelinka:2004} and spiral
pathway~\cite{Lavoue:2011}. The shapes of region boundaries defined by
the above methods are mostly restricted to regular formats, and such
nonadaptive neighborhoods cannot precisely reflect the local
geometrical or topological distortions. Region-wise descriptors can be
roughly divided into two categories: point-based and region-based
methods. Point-based methods provide the quantitative measure by
organizing single-valued point signatures into certain kinds of
distributions. Various kinds of point descriptors have been
incorporated in this manner, e.g., the shape index
(SI)~\cite{Toldo:2009}, shape diameter function
(SDF)~\cite{Shapira:2010}, heat kernel signature
(HKS)~\cite{Bronstein:2013}, Zernike
moments~\cite{Maximo:2011:RRI:2027471}, etc. Region-based methods
analyze the entire focal region through spectral
decomposition~\cite{Hu2009,JiangZZ:13}, which can robustly depict the
intrinsic geometry. However, due to the instability of local Laplacian
decomposition, these methods usually cannot well handle small and
complex regions.

\textbf{Region-based Partial Matching and Correspondence.} In
literature, region analysis has been discussed mainly in the research
of partial matching problems, and several categories of techniques
have been employed. Skeletal-graph-based approaches such
as~\cite{Biasotti:2006} couple geometry and structure in a single
skeletal descriptor based on the theory of Reeb graph. The main
drawback is that sub-parts cannot be recognized automatically.
Multi-criterion optimization
approaches~\cite{Bronstein:2008,Castellani:2008,Lavoue:2011} try to
match subparts by striking balance between significance and
similarity criteria. This type of methods require the knowledge of
correspondence between shapes, otherwise, it can only be solved by
alternating between correspondence and part area, which is
time-consuming. Bag-of-words (BoWs) technique has been
adopted~\cite{Zaharescu2009,Bronstein:2010:CVPR} to represent a shape or a
subpart as a collection of local feature signatures quantized in some
vocabularies of ``geometric words''. If the geometric vocabulary is
sufficient and the shapes have significant common parts, it is
possible to compare partially-similar shapes, otherwise, these
methods oftentimes fail to function properly. Furthermore, improper
additions of the spatial information and imprecise binning process
may lead to the averaging-off effects of geometric information. The
concept of non-point-wise correspondence was first proposed
in~\cite{Bronstein:2013} by using region-wise local descriptors and
optimizing over the integration domain upon which the integral
descriptors of the two parts match. This method can exactly match
fragments to entire shapes, however, since it utilizes the absolute
values in calculation like integration, it cannot deal with the
partially similar correspondence. Besides, many of the above
approaches rely heavily on exact or meaningful shape decomposition
process as in~\cite{Itskovich2011,Lavoue:2012}, which is
computationally expensive and significantly influences the final
correspondence results. Also, in order to achieve meaningful results,
many approaches utilize time-consuming post-processing like
in~\cite{Gal2006} and~\cite{Itskovich2011}. So it requires more
effective and generalized region detection techniques to help speed
up and improve the precision of the partial matching and
correspondence processes.

\begin{figure*}
\begin{center}
\includegraphics[width=0.68\textwidth]{Fig_2}
\end{center}
\caption[Illustration of the steps of our feature detection framework.]
  {The functional pipeline of our feature detection framework.
  (a)-(c) show the user's inputs in the specification process, red
  arrows in (a) and (b) denote the specified point of interest and
  contour scope, respectively. The bottom row shows the specified
  shape feature (e) and analogous feature regions (d) (we only
  display three cases here).}
\label{specification}
\end{figure*}

\textbf{Spectral Graph Wavelets.} Wavelet, which can localize a given
function both in space and in scale, is a powerful analytical tool in
signal processing~\cite{Mallat2008}. Unlike Fourier transform which
are globally defined, wavelet analysis is able to perform localized
multi-resolution analysis. Classical wavelets are constructed by
translating and scaling a mother wavelet in Euclidean space. However,
transplanting wavelets to graphs (specifically, triangular meshes) is
not straightforward due in part to the fact that it is unclear how to
apply the scaling operation on a signal that is defined on the mesh
vertices, so early studies using wavelet mostly relied on the
parameterization~\cite{Werghi:2002,Liu:2007}. One way to construct
wavelets on graph structures is using a diffusion operator and its
dyadic powers to obtain multi-scale wavelet and scaling
functions~\cite{Coifman2006,Hou2013}. Another approach to define
graph wavelets is by expanding graph functions with the eigen-system
of the graph Laplacian matrix and performing scaling in the frequency
domain~\cite{Hammond2011,Hou2012}, giving rise to the spectral
graph wavelet transform (SGWT). More recently, Kim et
al.~\cite{Kim:2012,Kim:2014} introduced a wavelet-based multi-scale
descriptor for the analysis of cortical surface signals using the SGWT
and Li et al.~\cite{Li:2013} proposed a SGWT-based descriptor and
utilized the intrinsic spatial pyramid matching (ISPM) for global
shape retrieval. Though these researches discover the potentials of
SGWs, they all concentrate on global shape analysis based on point
signatures, ignoring the SGWs' power in integrating the
local-to-global/in-between geometrical and topological information.
These inspire us to combine SGWs and bi-harmonic distance field to
enable the multi-scale and multi-level description of the region of
interests.

\section{Local-to-global Shape Feature Definition}
\label{sec:GF}

In this section, we introduce the definition process of the proposed
novel shape feature, which generalizes conventional features (e.g.,
point, line, or patch features) to a local-to-global level via user
specification. Our shape feature is extracted via the bi-harmonic
distance field~\cite{Lipman:2010}, which is robust, globally
``shape-aware'', parameter-free, and widely used in geometry
processing. Among these attractive properties, the consecutive
depicting power inspires us to incorporate it for integrating
multi-scale regional information that is required for subsequent
analysis. In addition, the cross-sections of contours in the
bi-harmonic distance field naturally form boundaries for
user-specified features, thus avoiding the shape decomposition
process.

Our shape feature is a user-specified partial region with two
parameters: the point of interest and the contour scope, both of
which can be determined using a simple user-interactive process. The
point of interest is the relative center of the feature, it can either
be picked directly on the mesh, as shown in
Fig.~\ref{specification}(a), or be automatically initialized as the
extreme point of a function defined on the surface. In order to specify
the contour scope of the point of interest, we should first introduce
the metric with which we set the scope, namely, the bi-harmonic
distance field.

Let us consider a 3D mesh represented as a graph $M = (V;E)$ with
vertices $V$ and edges $E$, where $V = \{v_1,v_2,...,v_n\}$ and $n$ is
the number of the vertices. A vector-valued function $f \colon
\textbf{V} \rightarrow \mathbb{R}^q $ defined on $V$ can be
represented as an $n \times q$ matrix, where the $i$-th row represents
the function value at $v_i$, we denote it as $f(i)$. According
to~\cite{Lipman:2010}, the bi-harmonic distance between vertex $v_i$
and $v_j$ can be expressed as
\begin{equation}
\label{eq:bd}
D_{bh}(i,j)^2=\sum_{k=1}^{m}\frac{(\chi_k(i)-\chi_k(j))^2}{\lambda_k^2},
\end{equation}
where $\{\lambda_k\}$ and $\{\chi_k(\cdot)\}$ are, respectively, the
first $m$ non-zero eigenvalues and the corresponding eigenfunctions of
the Laplacian-Beltrami operator with ``cotangent formula''
discretization~\cite{Meyer2003}.

For each vertex, Eq.~(\ref{eq:bd}) defines a diffusion field around
it. We compute the diffusion field of the specified interest point
(denoted as $v_s$) and then construct a set of contours
(Fig.~\ref{specification}(b)) across the entire model with contour
points located on the edges of the mesh model. These contours can very
well reflect the changes locally around the central point and
characterize the corresponding global structures, and these will be
discussed later. In order to make the setting of parameters more
stable, we normalize the original models using a unit box. Then the
total number of the contours distributed in the diffusion field can be
set empirically to the integer nearest to $max(D_{bh}(s,\cdot))/0.05$,
which is dense enough to depict the diffusion field. Then the user can
choose one of the contours to set the scope of the feature (as shown
in Fig.~\ref{specification}(b)) and denote it as $S_{s}$.
Fig.~\ref{specification}(c) illustrates the feature defined with the
red boundary.

Our shape features can be located anywhere and vary spatially in scale
depending on specific applications. They integrate the local and global
geometry, which affords their great potential in bridging differential
and integral geometry information.

\section{Multi-level And Multi-scale Shape Description}
\label{sec:Des}

Our novel descriptor primarily exploits the SGWs and the metric
statistics. This section introduces the SGWs-related description and
the contour-based statistics before the complete descriptor can be
constructed.

\subsection{SGW-based Description}

SGWT is introduced in~\cite{Hammond2011} for wavelet analysis
on graphs. The central idea is to define the required scaling in the
Fourier domain instead of the spatial domain. SGWT is determined by
the generating kernel $g \colon \mathbb{R} \rightarrow \mathbb{R}$. To
act as a band-pass filter, the kernel $g$ should satisfy $g(0) = 0$
and $\lim_{x \rightarrow \infty} g(x) = 0$. The SGWs are then
expressed as bivariate kernel functions
expanded on the manifold harmonic basis, which are formed by the
aforementioned Laplacian-Beltrami eigenfunctions on mesh models.
\begin{equation}
\label{eq:SGW}
\Psi_{t}(i,j)=\sum_{k=0}^{n-1}g(t\lambda_k) \chi_k(i)\chi_k(j),
\end{equation}
where $g$ is the real-valued wavelet generating kernel and $t$ is the
scale parameter. The $i$-th row of $\Psi_{t}(\cdot,\cdot)$
\begin{equation}
\label{eq:SGW_vert}
\psi_{t,i}(\cdot) = \Psi_{t}(i,\cdot)=\sum_{k=0}^{n-1}g(t\lambda_k) \chi_k(i)\chi_k(\cdot),
\end{equation}
is the spectral wavelet spatially-localized at $v_i$, and in the
frequency domain, localized at scale $t$. Fig.~\ref{SGW} shows the
SGWs of different scales located on one index point on the wolf model.
It may be noted that, the values of wavelets are attenuated and
oscillating on the mesh, and wavelets with a larger scale have a wider
oscillating window than those with smaller scales. Here it should be emphasized
that we choose to use the geometric mesh Laplacian instead of the combinatorial
Laplacian as originally applied in~\cite{Hammond2011} for more precise description
of the mesh geometry.

\begin{figure}[!to]
\begin{center}
\includegraphics[width=0.9\linewidth]{Fig_3}
\end{center}
\caption[Spectral graph wavelets on the wolf model.]
  {Spectral Graph Wavelets centered at one vertex on the wolf
  model. From left to right are wavelets from high frequency to low
  frequency with scale 1, 3 and 5.}
\label{SGW}
\end{figure}

It has been proved that SGWs can well represent the high frequency
and low frequency geometric information around the index
point~\cite{Hammond2011}.  Suppose we compute the spectral wavelets
at $J$ different scales $\{t_1,t_2,...,t_J\}$, and adopt the same
formulation of generating kernel functions used
in~\cite{Hammond2011}, given by
\begin{equation}
g(x)=\left\{
\begin{array}{lcl}
x^2                &      & {if\  x<1}\\
-5 +11x-6x^2 +x^3  &      & {if\  1 \leq x \leq 2}\\
4x^{-2}            &      & {if\  x>2}\\
\end{array} \right.,
\end{equation}
and the $J$ scales are selected to be logarithmically equally spaced
between the minimum scale $t_J = 2/\lambda_{max}$ and the maximum
scale $t_1=40/\lambda_{max}$, where $\lambda_{max}$ is the upper bound
of the Laplacian eigenvalues. The settings of $t_1$ and $t_J$
guarantee that $g(t_1 x)$ has power-law decay for $x>\lambda_{min}$
and $g(t_J x)$ has monotonic polynomial behavior for $x<\lambda_{max}$.

Using the above formulations, we can easily get the wavelet
coefficients of a given function on a specific vertex $v_i$ as
\begin{equation}
\label{eq:WMD}
W_f(t,i) = \langle \psi_{t,i},f \rangle = \sum_{l=0}^{n-1}g(t\lambda_l)\hat{f}(l)\chi_l(i),
\end{equation}
where
\begin{equation}
  \hat{f}(l) = \langle \chi_l,f \rangle = \sum_{i=0}^{n-1} \chi_l(i)f(i),
\end{equation}
and
\begin{equation}
  f(i) = \sum_{l=0}^{n-1} \hat{f}(l) \chi_l(i).
\end{equation}
Here, the signal function $f$ can be any kind of surface signal
depending on applications. For example, mean curvature,
characterizing detailed distortions, is a good choice for the
recognition of repetitive features within certain models, the
detection of similar features among models with different poses, etc.
For coarse-to-fine recognition, the HKS is more favorable thanks to
its robustness to noise. Different settings of $f$ will be shown
later for different applications. The coefficients $W_f$
obtained from the transformation is the inner product of the signal
function and the corresponding wavelet at scale $t$ and location $i$.
It is a representation of the signal for that scale, that is to say,
it describes the original signal in certain frequency with respect to
the local geometry and topology. Repeating this process for $J$
scales (as shown in Fig.~\ref{wmd}(a)-(d) with 4 scales), the set of
coefficients obtained comprises our multi-level descriptor.

\begin{figure}[!to]
\begin{center}
\includegraphics[width=0.9\linewidth]{Fig_4}
\end{center}
\caption[Statistics on wavelet decomposition coefficients.]
  {Statistics on signal's wavelet decomposition coefficients
  ($W_f(\cdot,t)$). (a)-(d) list $t$ from small ($t_1$) to large
  ($t_4$) value, corresponding to the decomposed signals varying from
  high to low frequencies. (e) illustrates the composition of
  statistics based on $W_f$ within one feature region.}
\label{wmd}
\end{figure}

Instead of using $W_f$ directly as the descriptor like
in~\cite{Li:2013} and other harmonic analysis researches, we
incorporate it into our descriptor by taking advantage of the
consecutive depicting power of the bi-harmonic distance field. In
order to comprehensively describe the geometric information contained
within the feature, we subdivide the feature region into thinner bands
with more contours as implemented in Section~\ref{sec:GF}, just as
shown in Fig.~\ref{specification}(c)-(e). The number of dense contours
may be set automatically as the integer nearest to $S_{s}/0.025$, and
such dense contours have been verified to be able to elaborately
depict and organize the inner geometrical information of the feature.
Fig.~\ref{wmd}(e) illustrates the setup of our SGW-based statistical
bands based on $W_f$, where different layers convey multi-level (from
high to low frequency) information and different bands (denoted in
yellow and green) in-between contours encode multi-scale information.
By stretching the ``matrix-like" statistic (as in Fig.~\ref{wmd}(e))
into a high dimensional vector, we obtain the descriptor of $v_s$,
denoted as $D_s$, given by
\begin{equation}
\label{eq:Des}
\hspace{-1.2mm}
D_s=[\underline{W_{t_1}^{b_1},...\ ,W_{t_1}^{b_L}},\\
\underline{W_{t_2}^{b_1},...\ ,W_{t_2}^{b_L}},\\
\underline{...}\ , \\
\underline{W_{t_J}^{b_1},...\ ,W_{t_J}^{b_L}}],
\end{equation}
where $W_{t_i}^{b_j}$ denotes the statistic of $W_f$ with scale $t_i$
on the $j$-th band, and $L$ is the number of contours. Here
$W_{t_i}^{b_j}$ can be expressed as the 1-norm of $W_f(t_i, \cdot)$
over the $j$-th band
\begin{equation}
\label{eq:WMD_sta}
W_{t_i}^{b_j} = \sum_{p \in b_j}| W_f(t_i,p)|,
\end{equation}
where $p$ is the vertex index, $p \in b_j$ denotes $p$ is a vertex
located on the $j$-th band.

In Fig.~\ref{wmd}, we observe that information contained in $W_f$ of
different time scales corresponds to the fine-to-coarse multi-level
information, and the statistics on the bands convey the near-to-far
multi-scale knowledge. These collectively make use of SGWs' power in
integrating geometric information. In addition, we notice that
descriptors based on $W_f$ are scale-invariant as long as SGWs are
normalized.

\subsection{Contour-based Multi-scale Statistics}

\begin{figure}[!to]
\begin{center}
\includegraphics[width=0.8\linewidth]{Fig_5}
\end{center}
\caption[Contour-based distance distribution.]
  {Contour-based distance distribution. The zoomed-in part
  shows the distances under current consideration when performing
  statistical calculation.}
\label{Descriptor}
\end{figure}

The contours of bi-harmonic distance field encode rich information of
local-to-global geometric variation. So we further introduce the
perimeters of contours and the distance distribution of contour points
to help characterize the focal region's shape in an orderly and
quantitative manner.

For a specified feature and its corresponding contours, we first
calculate contours' perimeters and concatenate them as
$\{p^{c_1},p^{c_2},...,p^{c_L} \}$, where $c_i$ denotes the index of
the $i$-th contour. Then, for each contour, we compute the Euclidean
distances between the contour points and their barycenter (as shown in
Fig.~\ref{Descriptor}), and further evaluate the probability
distribution of the distances. We denote the distance-related
statistics as $\{ds^{c_1},ds^{c_2},...,ds^{c_L} \}$. Here, $ds^{c_i}$
is a vector stacking up the probability distribution of the distances
concerning the $i$-th contour. We uniformly separate the distance
values into $M$ bins, ranging from zero to the maximum value after
removing the top and bottom 5\% to rule out possible outliers. Then
its stack pattern is
\begin{equation}
  ds^{c_i}=[\frac{num(b_1)}{num(c_i)},\frac{num(b_2)}{num(c_i)},...,\frac{num(b_M)}{num(c_i)}],
\end{equation}
where $num(b_j)$ is the number of points with distance values falling
in the $j$-th bin and $num(c_i)$ is the number of points on the $i$-th
contour. For multiple contours with the same value, they should be
considered as a whole when performing statistical calculation.

These two measurements help describe the shape of the bi-harmonic
distance field completely and identify the details of shape's
distortion. The purpose of introducing the distance distribution is
to distinguish between different contours with the same perimeters.

\subsection{Informative Region-based Descriptor}

So far, the separate parts of our descriptor have all been introduced.
Now, it sets the stage for us to integrate them together to form our
final informative hi-dimensional descriptor as
\begin{equation}
\label{eq:desFinal}
\begin{aligned}
  \hspace{-0.5mm}
  D_s =  [\omega_s*(W_{t_1}^{b_1},...\ ,W_{t_J}^{b_L}),\\
  \omega_p*(p^{c_1},...\ ,p^{c_L}),\\
  (ds^{c_1},...\ ,ds^{c_L})],
\end{aligned}
\end{equation}
where $\omega_s$ and $\omega_p$ are weights to adjust the
contributions of the three parts in the descriptor. The settings of
these weights will be detailed in Section~\ref{sec:Exp}.

Our descriptor integrates the attractive properties of both SGWs and
contour-based measurements. From the viewpoint of feature mapping,
SGWs establish a powerful foundation for hierarchical representation
of the geometrical and topological details. Our design of the
regional description encodes the shape feature in the aspects of both
``breadth'' and ``depth'', paving the way for our feature detection
framework.

\section{Feature Detection Framework}
\label{sec:Framework}

Using the same way to define feature regions around candidate points
and formulate the corresponding descriptions, descriptors concerning
candidate regions on the same model or different models in the
database can be easily computed for analytical purposes.

\subsection{Constructing Descriptors over Shapes}

To construct descriptors on candidate regions on one or more models,
the central points and contour scopes should also be determined first.
As for the central points, we implement the farthest-point-sampling
strategy~\cite{Moenning:2003} to uniformly extract points on the mesh
model (as shown in Fig.~\ref{Sampling}). This strategy ensures not
only the uniform distribution of the candidate points, but also the
inclusion of the end points, which are interesting alternatives for
user's selection. However, we want to mention that the sampling
process is optional, and users could either pick the desired sampling
methods or simply use all the vertices as candidates according to
specific applications.

Then the construction of descriptors across versatile shapes is based
on the knowledge of the shape feature defined. That is, candidate
regions are determined automatically according to the contour scope of
the shape feature specified. Suppose that $v_{s}$ and $v_{i}$ are
respectively the interest point and one of the candidate points. Then
the scope of $v_{i}$ is set as
$S_{s}*max(D_{bh}(i,\cdot))/max(D_{bh}(s,\cdot))$ and this can help
ensure the robustness of our method for deformable models.  With the
region scope determined, the construction of the corresponding
descriptor is conducted in the same way as the specified feature (in
Section~\ref{sec:Des}), which is detailed in
Algorithm~\ref{alg:Framwork}.

\begin{figure}[!to]
\begin{center}
\includegraphics[width=0.9\linewidth]{Fig_6}
\end{center}
\caption[Point sampling on the horse and Santa models.]
  {Sampling results on horse and Santa models using the
  farthest-point-sampling strategy. Only part of the sampling points
  are shown for visual clarity.}
\label{Sampling}
\end{figure}

With each candidate region equipped with a high-dimensional
descriptor, our key task is to analyze the similarity in the
descriptor space. We shall first introduce the proper measurements
for this new space. It has been found that both $L1$ and $L2$ norms
are discriminative enough to measure the distance between any two
descriptors, representing the corresponding regions. As alternatives,
the covariance distance and $\chi{2}$ distance are also tested to be
good choices for comparing the distributions' similarities.

\subsection{Feature Detection and Framework Properties}

Here, we detail the effectiveness of our feature detection framework
together with several of its attractive properties and more results
will be shown in Section~\ref{sec:Exp}. It shall first be emphasized
that high sampling rates always lead to dense distribution of the
candidate points, thus several neighboring points may have similar
diffusion regions, and this will lead to multiple detected results
that are in the vicinity of each other. Therefore we empirically
reject candidate regions that have more than $50\%$ overlaying with
the afore-ranked regions. This strategy can ensure the uniqueness of
the detected features as well as the broader coverage of all relevant
feature regions, and also make our approach robust to different
sampling processes.

\begin{figure}[!to]
%\begin{center}
\includegraphics[width=0.85\linewidth]{Fig_7}
%\end{center}
\caption[Detection of repetitive features with different scales.]
  {Detection of repetitive features with different scales. The
  first row shows the specified features and the second row shows the
  detected results within the bear model. Yellow points on query
  models denote the specified points of interest.}
\label{scales}
\end{figure}


\begin{algorithm}
\caption{Construction of Descriptors over Shapes.}
\label{alg:Framwork}
\begin{algorithmic}[1]
\REQUIRE User-specified interest point $v_{s}$ and contour index $n_{s}$;
\ENSURE  Descriptors of the specified feature ($D_{s}$) and all
         the other candidate regions ($\{D_{i}\}$);
\STATE Conduct the eigen-decomposition on all models;
\STATE Compute the $W_f$ of each point using Eq.~(\ref{eq:WMD});
\STATE Compute the bi-harmonic distance of $v_{s}$, denoted as $D_{bh}\{s,\cdot\}$;
\STATE Uniformly construct  $max(D_{bh}(s,\cdot))/0.05$ contours across the entire model;
\STATE With user's specification of the contour number ($n_{s}$), compute the corresponding value as $S_{s}$;
\STATE For any candidate point $v_{i}$, compute its bi-harmonic distance $D_{bh}\{s,\cdot\}$,
       and set its region scope as $S_{s}*max(D_{bh}\{i,\cdot\})/max(D_{bh}\{s,\cdot\})$;
\STATE Reconstruct $S_{s}/0.025$ contours on the specified feature and all the candidate feature regions;
\STATE Compute the SGWs-based and contour-related measurements using Eq.~(\ref{eq:desFinal});
\STATE Return $D_s$ and $\{D_{i}\}$.
\end{algorithmic}
\end{algorithm}

We first show a simple feature detection result within the bear model
as displayed in Fig.~\ref{scales} (here, mean curvature is chosen as
the signal function). The top row shows the features specified with
different scales and the second row shows the detected results. It can
be observed that the detected similar feature regions are affected by
the specification of the feature scope (the selected contour indices
here are 2, 3, and 6, respectively). Though small-scale query leads to
rather trivial outcomes, the most similar parts are still among the
top-ranked results. Furthermore, it is obvious that larger scales can
lead to more accurate results thanks to added information. The
analogous feature regions defined (proportional to the specified
feature) and the corresponding descriptors jointly ensure the accuracy
of the detection of the repetitive feature regions.

Our shape feature and its description possess many attractive
properties like being concise to store, fast to compute, and efficient
to match, etc. Here, we demonstrate two more desirable properties that
can facilitate various practical tasks.

\textbf{Isometry-invariance.} We verify the property of
isometry-invariance through tests carried on three categories of
models in different poses. These models are chosen from the SCAPE and
TOSCA databases. As visualized in Fig.~\ref{pose-human}, we
deliberately specify features on the human arm containing the elbow,
dog leg with elbow and finger with knuckle to validate the property.
The models in Fig.~\ref{pose-human}(b) are the deformed ones showing
the top-2 most similar feature regions detected on each of them (red
and orange highlight the first and second one, respectively). The
top-2 detected results are shown here since our approach is capable
of identifying similar regions, but distinguishing between symmetric
parts is beyond the technical scope of this work. The retrieved
results empirically prove that our region-based descriptor is
isometry-invariant, which is inherited from the properties of graph
wavelets and bi-harmonic distance field.

\begin{figure}[!to]
%\begin{center}
\includegraphics[width=1\linewidth]{Fig_8}
%\end{center}
\caption[Illustration of the isometry-invariance property.]
  {Illustration of isometry-invariance. (a) Query. (b) Detected
  results on deformed models with red and orange denoting the top-2
  similar regions.}
\label{pose-human}
\end{figure}

\textbf{Robustness to Noise.} We add 0.5 (of the mesh's mean edge
length) noise to the query models as shown in Fig.~\ref{noise}(a) and
set the signal as HKS. We selectively enlarge the SGWs-related part
of our descriptor to combat noise. The features are specified on these
noisy models as shown in Fig.~\ref{noise}(b). From the detection
results displayed in Fig.~\ref{noise}(c), it can be observed that even
with large noise, our approach can still recognize the intrinsic
geometric characteristics of the features and identify the
corresponding similar features, including the ball shape on the elk
model and the bended handle of the kettle model from the shape
database, thanks to the robustness and stability of bi-harmonic
distance field and HKS. The robustness of our approach is of great
significance in practical applications as will be demonstrated in
Section~\ref{sec:Exp}.

\section{Experimental Results and Discussions}
\label{sec:Exp}

\begin{figure}[!to]
%\begin{center}
\includegraphics[width=0.95\linewidth]{Fig_9}
%\end{center}
\caption[Coarse-to-fine recognition on elk and kettle models.]
  {Coarse-to-fine recognition on elk and kettle models with 0.5
  (mean edge length) noise. (a) Noisy model. (b) Specified features.
  (c) Recognized regions.}
\label{noise}
\end{figure}

\begin{table}
\centering
\caption{Run Time for Constructing Descriptors}
  \label{table:time}
  \renewcommand{\arraystretch}{1.3}
\begin{tabular}{p{0.18\textwidth}p{0.18\textwidth}p{0.12\textwidth}p{0.12\textwidth}p{0.12\textwidth}}
\hline
\multirow{2}{*}{Model} &
\multirow{2}{*}{\# vertices} &
\multicolumn{3}{c}{Timing (min)} \\
\cline{3-5}
&\multicolumn{1}{c}{} & Eigen &  $W_f$ & Total\\
\hline
Dinosaur   & 14K & 0.17 & 0.14 & 0.33 \\
Dog       & 26K & 0.30 & 0.35 & 0.69 \\
Armadillo & 34K & 0.38 & 0.48 & 0.97 \\
Santa     & 75K & 0.93 & 1.23 & 2.32\\
Dragon & 430K & 5.51 & 8.16 & 14.15\\
\hline
\end{tabular}
\end{table}

In this section, we demonstrate the performance of our approach via
experiments in various aspects. All the experiments were conducted on
a 3.5GHz Intel(R) Core(TM) i7 computer with 16GB memory. We used the
cusparse and cublas libraries in CUDA to help reduce the computational
time for wavelet transform significantly. For instance, for the SHREC
2007 partial retrieval dataset, in which the average model size is 18K
vertices, the whole process of constructing descriptors on one model
takes an average of 0.38 minutes, and the whole database costs 114
minutes with $20\%$ sampling rate for each model. More timing details
concerning versatile models are shown in Table~\ref{table:time}.

\subsection{Parameter Evaluation}

There are several parameters in our approach, most of them can be set
automatically or empirically set as constants. Two parameters, the
sampling rate and the signal function, could be tuned according to
specific applications' requirements for better results.

\textbf{Parameters in the Construction of Descriptors.} For the
SGWs-related part of the descriptor, we calculate bi-harmonic
distances and graph wavelets using the first 300 eigenvalues of each
model, and they only need to be computed once. The number of graph
wavelets' time scales is set to be 5, which can well represent the
frequency information. We empirically set the number of the bins in
distance distribution to be 10, and it has been tested to be
sufficient to characterize the structure features. The weights
$\omega_s$ and $\omega_p$ are automatically set (based on the mean
scale of each part) to balance the three parts that comprise the
descriptor. Fig.~\ref{ws} demonstrates the function of graph
wavelets by comparing the retrieval results with $ws = 0$ and
$w_s \neq 0$ on SHREC database and the corresponding constructed
database (note that each model is added with 0.5 mean-edge-length
noise as shown in Fig.~\ref{ws}(a)). The relevant number is the
number of retrieved models that contain partial regions that match
the query region (here, ant head and plane tail). It clearly shows
that graph wavelets holds the power to characterize the focus regions
discriminatively and robustly. Therefore, we selectively enlarge the
weight of SGWs-related part ($w_s$) to combat the disadvantages of
the other parts in specific applications.

\begin{figure}[!to]
%\begin{center}
\includegraphics[width=0.9\linewidth]{Fig_10}
%\end{center}
\caption[Retrieval results on the SHREC database.]
    {Retrieval results (on the SHREC database) with $ws = 0$ (red line)
    and $w_s \neq 0$ (blue line), and dash lines show the results on
    noisy models.}
\label{ws}
\end{figure}

\textbf{Sampling Rate.} The setting of sampling rate depends on
specific applications. Even when the sampling rate is decreased to
$5\%$, our sampling strategy still ensure the inclusion of endpoints.
It should be noted that if the application requires high-precision
detection, all the vertices of the model could be taken into
consideration. For the feature detection framework, the sampling rate
of $20\%$ can meet almost all the needs in our experiments.

\textbf{Signal Selection.} The signal function influences SGW-related
statistics directly. Therefore, signal selection is among the key
problems that should be considered. In principle, the selection
depends on specific applications as analyzed in Section~\ref{sec:Des}.
Furthermore, some applications require the signal to be intrinsic, for
which signals like Gaussian curvature, thickness, etc. are expected to
perform better.

\subsection{Repetitive Feature Detection within Certain Model}

Repetitive feature detection is of great importance to applications,
such as self-symmetry detection~\cite{Gal2006} and non-local
processing propagation~\cite{Maximo:2011:RRI:2027471}. We randomly
select some regions of interest to be the shape features as
illustrated in Fig.~\ref{detect_self} (in red). The specified features
show that our diffusion-manner demarcation can well cover the interest
regions of any kind of shapes if only the interest points and scopes
are chosen properly. The detection results show that our descriptor
can reliably locate the repetitive feature regions within the dragon
model even with large deformations, and it can effectively distinguish
between the lumpy local shape of dinosaur's tail from its cylinder-like
legs, etc. It very well demonstrates that our SGW-centered descriptor
can depict the local details and reflect multi-scale geometric
distortions thanks to its gradational construction.

\begin{figure}[!to]
\begin{center}
\includegraphics[width=0.9\linewidth]{Fig_11}
\end{center}
\caption[Repetitive feature detection within one model.]
  {Repetitive feature detection within one model. Red
  denotes query feature and blue denotes some detected
  results.}
\label{detect_self}
\end{figure}

\subsection{Feature Detection in Database}

The performance of our feature detection framework is evaluated on
the SHREC 2007 watertight retrieval benchmark, which contains 20
categories and each consists of 20 meshes. In order to demonstrate
that our framework is not restricted to the segmentation-based regions
like four-legs, we test three kinds of artificial models and each of
the query model is unique without any transformed equivalent in the
dataset.

We deliberately specify features on kettle, glasses, and chair as
shown in Fig.~\ref{detect_database_1}(a). The slab of the chair is a
vivid example, and the existing methods based on segmentation cannot
achieve such trans-boundary shapes. It's obvious that the top-4
similar parts in Fig.~\ref{detect_database_1}(b) resemble the query
region very well, which validates that our descriptor based on the
continuously distributed contours could characterize the focal region
thoroughly. The feature detection process has the ability to correctly
identify the most similar parts with equal-proportional scales thanks
to the stable characteristic of bi-harmonic distance. By examing which
models in the database contain the similar shapes such as kettle base,
eyeglasses or slab of chair, our framework can facilitate co-analysis
across models and enable search-based shape modeling.


\begin{figure}[!to]
%\begin{center}
\includegraphics[width=0.95\linewidth]{Fig_12}
%\end{center}
\caption[Feature detection on the SHREC 2007 watertight retrieval
  database.]
  {Feature detection on the SHREC 2007 watertight retrieval
  database.}
\label{detect_database_1}
\end{figure}

\subsection{Comparisons and Discussion}
\label{sec:App}

Due to the unique technical strategy of our feature description and
detection, there is no existing work that is directly comparable with
ours. From the perspective of applications, partial matching appears
to be the most relevant one. Therefore, we first compare our approach
with five popular existing methods in partial matching or shape
retrieval based on region description. These methods can all properly
fit into our framework and make effective comparisons. They are
\begin{itemize}
\item D2 Shape Distribution (D2): statistics of distances between any
      random pair of points on the region~\cite{Osada:2002}.
\item Conformal Factors (CF): statistics of conformal geometric factors
      of points on the region~\cite{Ben-Chen:2008}.
\item Zernike Moments based signature (ZM): local shape signature based
      on transformation of Zernike Moments~\cite{Maximo:2011:RRI:2027471}.
\item Local SDF Signature (SDF): statistics of points' Shape Diameter
      Function~\cite{Shapira:2010}.
\item Patch Spectral Geometric Features (PS): normalized spectra of patch
      spectrum decomposition~\cite{Hu2009}.
\end{itemize}
The above methods induce the direct geometric measurement (D2), intrinsic
curvature-related measurement (CF), heightmap-based measurement (ZM),
volume-based measurement (SDF), and spectral-analysis measurement (PS),
respectively.

\begin{figure}[!to]
\includegraphics[width=1\linewidth]{Fig_13}
\caption[Precision plots of different detection methods.]
    {Precision plots of different detection methods. (a)-(c) show
    results of different queries. (d) shows the average P-R graph on SHREC
    watertight database.}
\label{PR}
\end{figure}

\begin{table}
\centering
\caption[Precisions of different detection methods.]
 {Precisions of different detection methods.}
  \label{table:precision}
  \renewcommand{\arraystretch}{1.3}
\begin{tabular}{p{0.14\textwidth}|p{0.08\textwidth} p{0.08\textwidth}|p{0.08\textwidth}p{0.08\textwidth}|p{0.08\textwidth}p{0.08\textwidth}}
\hline %\hline
\multirow{2}{*}{Method} &
\multicolumn{2}{c|}{Ant Head} &
\multicolumn{2}{c|}{Plane Tail} &
\multicolumn{2}{c}{Human Leg} \\
\cline{2-7}
& 15 & 30 & 15 & 30 & 30 & 60\\
\hline
Ideal & 100\% & 66.7\% & 100\% & 66.7\%  & 100\% & 66.7\% \\
D2 & 26.7\% & 30.0\% & 26.7\% & 23.3\%  & 40.0\% & 41.7\% \\
CF & 33.3\% & 30.0\% & 53.3\% & 56.7\%  & 46.7\% & 48.3\% \\
ZM & 53.3\% & 43.3\% & 53.3\% & 53.3\%  & 46.7\% & 50.0\% \\
SDF & 46.7\% & 56.7\% & 46.7\% & 50.0\% & 50.0\% & 58.3\% \\
PS & 80.0\% & 60.0\% & 46.7\% & 43.3\%  & 73.3\% & 60.0\% \\
\textbf{Ours} & \textbf{93.3\%} & \textbf{63.3\%} & \textbf{80.0\%} & \textbf{60.0\%}  & \textbf{86.7\%} & \textbf{63.3\%}\\
\hline
\end{tabular}
\end{table}

\begin{figure*}
\begin{center}
\includegraphics[width=0.95\linewidth]{Fig_14}
\end{center}
\caption[Comparisons of different detection methods for partial matching.]
 {Comparisons of different detection methods for partial
  matching. The queries are ant head, human leg, and plane tail.}
\label{comparison}
\end{figure*}

We conduct comparison tests on SHREC 2007 watertight database as well
as the McGill database that contains 255 objects divided into ten
classes and the intra-class variations consist of non-rigid transforms
applied to models. By setting proper parameters in the above methods,
effective results are achieved. Fig.~\ref{PR} shows the detection
results of all methods concerned. (a)-(c) are the detection results
with the queries of ant head, plane tail, and human leg.  In the SHREC
database, every category contains 20 different models, so we retrieve
the top 20 results of ant head and 40 for human leg, since left and
right legs of human are documented separately (we call it the
multiple-region case). (d) shows the average PR graph based on all
queries without multiple regions.

Table~\ref{table:precision} details the precisions of different
detection methods. The precision is computed as the ratio of relevant
number to the retrieved number (the second line shows the retrieved
number for each query).  Table~\ref{table:precision} and
Fig.~\ref{comparison} collectively show that each method has its own
strength in describing some specific kind of shape. D2 is excellent
in describing regular shapes, such as spheres, because the histogram
statistics sometimes have the averaging effect on the spatial
information, and the ant head with two tentacles makes it difficult
for D2 to depict. CF, which integrates the gaussian-curvature
knowledge, performs well in characterizing highly-curved region such as
the plane tails. ZM is suitable for depicting small patches, as for
large feature regions it works better in detecting highly curved ones.
SDF takes into account the volume-based information and the region's
area ratio knowledge, so it performs well for cylinder-like shape, but
not for local complex shapes. PS performs very well in most cases except
for the relatively small and complex structure like plane tails due to
the instability of regional Laplacian decomposition. In contrast, our
method performs stably with high precision in detecting various types
of feature regions across different models.

Moreover, we conduct the comparison with two most related works that
cannot fit into our feature detection framework, namely,
Gal's~\cite{Gal2006} and Itskovich's works~\cite{Itskovich2011}.
Since these two works obtain the feature regions by decomposition
process and clustering of the pre-divided patches, there inevitably
exist over-segmentation phenomenon as shown in the zoomed-in part of
Fig.~\ref{Gal}. In comparison, our method can flexibly specify the
interesting feature region and exactly detect the similar regions with
the proper scales.

\subsection{Applications}

Our feature detection framework enjoys plenty of desirable properties
as demonstrated in Section~\ref{sec:Framework}. They can further
facilitate a host of applications as will be shown and analyzed.

\textbf{Partial Matching and Restoration.} We analyze the combined
models from SHREC 2007's partial retrieval dataset, which comprises
the SHREC 2007's watertight dataset and a query set of 30 models.
Each combined model is obtained by merging or removing several
subparts of models belonging to the watertight dataset. Existing
algorithms concerning partial matching consider the retrieval of the
query set models as a big challenge.  We demonstrate that our
framework provides a powerful tool to restore the combined models,
thus can effectively aid the matching and recognition. As the three
cases shown in Fig.~\ref{partial-query}, choosing proper interest
points (yellow points in (a)) and contour scopes could give rise to
diffusion regions that cover large scale of the searched models.
After completing the detecting process in the watertight database,
the original model can be recognized as shown in (b), and (c) shows
another two detected models containing the top-ranked similar feature
regions. Moreover, if conventional point-wise correspondence is
required for downstream tasks, we can further achieve this goal
easily. Since our feature is defined in a diffusion manner, after the
similar region is matched, the exact point matching can be obtained
by shrinking the diffusion region back to its source point, thus
helping the conventional point-based correspondence and other complex
partial matching tasks.

\begin{figure}[!to]
\includegraphics[width=0.98\linewidth]{Fig_15}
\caption[Comparisons of detecting complex feature regions.]
  {Comparison on detection of complex feature regions. Figure
  $(a)$ and $(b)$ are cited from the corresponding works. On each
  model, the flower inside the yellow circle is the query region.}
\label{Gal}
\end{figure}

\textbf{Model Recognition.} Another immediate application is model
recognition based on key components as suggested
in~\cite{Sipiran:2012}. The feature detection framework is much more
powerful and does not need the complex process of finding key
components as in~\cite{Sipiran:2012}.  Fig.~\ref{recognition}(a) shows
the query models, on which we selectively specify three or four
features that are considered to be essential for characterizing the
armadillo and horse model. The whole shape similarity is computed as
the sum of the distances between the specified feature regions of the
query model and the corresponding regions of the target model.
Apparently, specifying more feature regions can achieve more precise
results. The ranked retrieval results in Fig.~\ref{recognition}(b)
show that our method can correctly retrieve the relevant models from
the database even when the models are somewhat incomplete (like the
first recognized armadillo model).

\begin{figure}[!to]
\includegraphics[width=0.95\linewidth]{Fig_16}
\caption[Partial shape matching and restoration.]
 {Partial matching and restoration on the SHREC's partial query
  dataset. (a) Combined models with query regions in red. (b) Restored
  models.  (c) Another two models containing the top-ranked similar
  feature regions.}
\label{partial-query}
\end{figure}

\textbf{Other Potential Applications.} Many more practical applications
could potentially benefit from the attractive properties of our
approach. For example, thanks to the robustness demonstrated in
Section~\ref{sec:Framework}, our framework can serve as the foundation
for search-based modeling and coarse-to-fine part replacement that
frequently relies upon the conventional denoising processes in the
pre-processing stage. The reliability demonstrated in the test of
repetitive feature detection shows that our detection results could
potentially help with the recognition of repeated patterns,
cut-and-paste editing, and self-symmetry detection.

\section{Chapter Summary}
\label{sec:feature:sum}

In this chapter, we have detailed the description and detection of our
proposed generalized local-to-global features on 3D geometric models,
which organically couple both local (differential) and global
(integral) geometric attributes. The multi-scale and multi-level
descriptor based on SGWs has exhibited its potential in depicting any
user-specified feature region and distinguishing among descriptor
vectors in the corresponding region-wise descriptor space.
Furthermore, our novel descriptor is comprising many desirable
properties which can facilitate a host of graphics applications, as
showcased in our comprehensive experiments. Extensive comparisons with
other state-of-the-art techniques/methods have demonstrated certain
key advantages of our method in terms of geometry-awareness,
reliability, robustness, etc.

\begin{figure}[!to]
\begin{center}
\includegraphics[width=1\linewidth]{Fig_17}
\end{center}
\caption[Model recognition by identifying multiple components.]
 {Model recognition by identifying multiple components of
 interest. (a) Query models with 3 or 4 specified features.
  (b) Top three recognized models.}
\label{recognition}
\end{figure}

One of our future work is to make the universal interface and functionality
more flexible and intuitive, allowing users to sketch the boundary of any
intended feature region. We also plan to to incorporate more powerful
analytical tools into the descriptor space for improved performance.
Finally, we intend to broaden the application scope of the proposed shape features to support feature-centric registration, structure-driven co-segmentation
and high-fidelity model production. 